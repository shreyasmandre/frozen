\documentclass[letterpaper,10pt,prl,twocolumn,aps]{revtex4-1}
\usepackage{fullpage}
\usepackage{amsmath}
\usepackage{amsfonts}
\usepackage{amssymb}
\usepackage{graphicx}
\usepackage{slashbox}
\usepackage{color}
\usepackage{longtable}
\usepackage{array}
\usepackage{dashrule}



\begin{document}

\title{Linear stability for transient base flows}
\author{Shreyas Mandre}
\affiliation{Brown University, Providence RI 02912 USA}
\author{Anja Slim}
\affiliation{Schlumberger-Doll Research Center, Cambridge MA 021?? USA}
\begin{abstract}
(i) We look at solutal convection as a prototypical problem for linear stability of transient base states. (ii) We derive a formulation to determine the amplification of the most dangerous perturbation indced by the slowest growing norm. (ii) This framework reduces to various classical and modern formulations of stability in special limits. (iv) We provide rigorous justification for the frozen cefficient analysis to determine the threshold time for the onset of instability.
\end{abstract}
\maketitle
The study of solutal convection is motivated by carbon dioxide sequestration in deep underground aquifers\cite{SlimRama10}. In this problem, convection ensues as an instability of a diffusing (and hence transient) solute concentration gradient. A common method used for determining the threshold time for the onset of convection is the frozen coefficient analysis. It is thought that this analysis requires the base state to change with time much slower than the perturbations grow (or decay), and therefore the analysis is invalid at the threshold of onset (because the perturbations neither grow nor decay). We present an alternative method for the stability analysis based on the growth of positivie definite norms quantifying the size of the perturbations. To address the sensitive dependence of the stability results on the choice of the norm, we extend the analysis to find the norm which leads to least possible amplification. Using this modification, we show rigorously that a separation of time-scales is not require for the validity of frozen coefficient analysis, and that it gives the correct threshold for the onset of instability. We demonstrate these for a special case of the onset of solutal convection in porous medium. (A few words about the utility of this method is other problems.)

Mathematical model for solutal convection

Brief rationale and description of the method, and comparison with existing methods.

Criteria for threshold and equivalence with frozen coefficient

Results from computations showing agreement, and teaser for future paper.

\bibliography{timevariant}
% \bibliographystyle{}
\end{document}
